\chapter{Аналитическая часть}
В этом разделе будут представлены описания алгоритмов умножения разреженных матриц, последовательного и параллельного.

\section{Последовательный алгоритм умножения разреженных матриц}

\textbf{Разреженная матрица} \cite{sheyker} — в численном анализе и научных вычислениях, разреженная матрица или разреженный массив представляет собой матрицу, в которой большинство элементов равны нулю. Не существует строгого определения того, сколько элементов должно быть нулевым, чтобы матрица считалась разреженной, но общий критерий состоит в том, что количество ненулевых элементов примерно равно количеству строк или столбцов. Напротив, если большинство элементов отличны от нуля, матрица считается плотной. Количество элементов с нулевым знаком, деленное на общее количество элементов, называют разреженностью матрицы.

При хранении разреженных матриц и манипулировании ими на компьютере необходимо использовать специализированные алгоритмы и структуры данных, которые используют разреженную структуру матрицы. Специализированные компьютеры были созданы для разреженных матриц, поскольку они распространены в области машинного обучения. Операции с использованием стандартных структур и алгоритмов с плотной матрицей медленны и неэффективны при применении к большим разреженным матрицам, поскольку обработка и память тратятся на нули. Разреженные данные по своей природе легче сжимать и, следовательно, требуют значительно меньше памяти. Некоторыми очень большими разреженными матрицами невозможно манипулировать с помощью стандартных алгоритмов плотных матриц.

Последовательный алгоритм умножения двух разреженных матриц является набором определенных шагов.
\begin{enumerate}
	\item Создать матрицу результата.
	\item Первый цикл проходит по строкам матриц.
	\item Второй цикл идет до тех пор, пока в массиве значений левой матрицы не закончится рассматриваемая строка. В этом цикле выполняются следующие действия:
	\begin{enumerate}
		\item Пройти по всем элементам правой матрицы и умножить их на элементы, рассматриваемой на данном шаге строки. Прибавить результат умножения к элементу массива промежуточных значений, номер элемент а определяем по значению столбца правой матрицы, в котором находится, рассматриваемый элемент.
		\item После умножения всех значений правой матрицы сохранить значение массива в вектор значений матрицы результата. А так же обновить вектор столбцов для каждого элемента.
		\item Добавить размер вектора столбцов в вектор строк матрицы результата.
	\end{enumerate}
	\item После завершения первого цикла в матрице результата находится результат умножения переданных двух матриц. Вернуть из метода матрицу ответа.
\end{enumerate}

\section{Параллельный алгоритм умножения разреженных матриц}

Параллельный алгоритм отличается от последовательного возвращаемым результатом. Из-за необходимости независимости работы потоков, результатом умножения двух разреженных матриц будет обычная матрица, так как в последовательном алгоритме используется добавление в вектор строк матрицы результата, в следствие чего в параллельном алгоритме произойдет путаница.
В параллельном алгоритме внешний цикл разбивается на несколько циклов, количество которых равно количеству потоков. Каждый поток выполняет свой цикл.

\section*{Вывод}

В данном разделе были рассмотрены принципы работы последовательного и параллельного алгоритмов умножения разреженных матриц.