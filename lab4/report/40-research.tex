\chapter{Исследовательская часть}

В данном разделе будут приведены примеры работы программ, постановка эксперимента и сравнительный анализ алгоритмов на основе полученных данных.

\section{Технические характеристики}

Технические характеристики устройства, на котором выполнялось тестирование:

\begin{itemize}
	\item операционная система: Manjaro xfce \cite{ubuntu} Linux \cite{linux} x86\_64;
	\item память: 8 Гб;
	\item мобильный процессор AMD Ryzen™ 7 3700U @ 2.3 ГГц \cite{intel}.
\end{itemize}

Тестирование проводилось на ноутбуке, включенном в сеть электропитания. Во время тестирования ноутбук был нагружен только встроенными приложениями окружения, а также непосредственно системой тестирования.

\section{Демонстрация работы программы}

На рисунке \ref{img:primer} представлен результат работы программы.

\img{120mm}{primer}{Пример работы программы}
\FloatBarrier

\section{Время выполнения реализации алгоритмов}

Время работы реализации алгоритмов измерялось при помощи функции chrono::high\_resolution\_clock::now() из библиотеки <chrono> языка С++.

Сравнение работы последовательного и параллельного алгоритмов проводилось при 16 потоках.

\begin{table}[h]
	\begin{center}
		\caption{Результаты замеров реализаций алгоритмов умножения разреженных матриц}
		\label{tbl:random}
		\begin{tabular}{|c|c|c|}
			\hline
			Размер & Последовательный & Параллельный \\
			\hline
			400 &  245643 & 76332\\
			\hline
			500 &  197758 & 58654\\
			\hline
			600 &   & \\
			\hline
			700 &   & \\
			\hline
			800 &   & \\
			\hline
			900 &   & \\
			\hline
			1000 &   & \\
			\hline
		\end{tabular}
	\end{center}
	
\end{table}
\FloatBarrier



\begin{table}[h]
	\begin{center}
		\caption{ Время выполнения реализации параллельного алгоритма умножения разреженных матриц (мкс)}
		\label{tbl:random}
		\begin{tabular}{|c|c|}
			\hline
			Количество потоков & Время \\
			\hline
			1 &  419012\\
			\hline
			2 &  261648\\
			\hline
			4 &  156537\\
			\hline
			8 &  131731\\
			\hline
			16 &  113116\\
			\hline
			32 &  116371\\
			\hline
			64 &  131898\\
			\hline
		\end{tabular}
	\end{center}
\end{table}
\FloatBarrier

\begin{figure}[h]
	\captionsetup{singlelinecheck = false, justification=centering}
	\centering
	\begin{tikzpicture}
		\begin{axis}[
			xlabel={количество потоков},
			ylabel={время, мкс},
			width=0.95\textwidth,
			height=0.3\textheight,
			xmin=0, xmax=65,
			legend pos=north west,
			xmajorgrids=true,
			grid style=dashed,
			]
			\addplot[
			color=blue,
			mark=asterisk
			]
			table [x=N, y=time]{
				N time
				1 419012
				2 261648
				4 156537
				8 131731
				16 113116
				32 116371
				64 131898
			};
		\end{axis}
		
	\end{tikzpicture}
	\caption{Сравнение времени работы параллельного алгоритма при разном количестве потоков}
	\label{time_best}
\end{figure}

\FloatBarrier

\section*{Вывод}

Реализация алгоритма сортировки подсчетом работает быстрее остальных двух во всех трех случаях, что произошло благодаря небольшому диапазону значений генерируемых массивов --- не более 1000. Это привело к линейному характеру зависимости трудоемкости данного алгоритма от размера массива и линейному от мощность диапазона значений в массиве.


