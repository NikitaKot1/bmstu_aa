\chapter{Технологическая часть}

В данном разделе будут приведены требования к программному обеспечению, средства реализации и листинги кода.

\section{Требования к ПО}

К программе предъявляется ряд требований:
\begin{itemize}
	\item на вход подаётся массив сравнимых элементов (целые числа);
	\item на выходе — тот же массив, но в отсортированном порядке.
\end{itemize}

\section{Средства реализации}

В качестве языка программирования для реализации данной лабораторной работы был выбран язык программирования Си++ \cite{pythonlang}. 

В данном языке есть все требующиеся для данной лабораторной инструменты разработки. 

Процессорное время работы реализаций алгоритмов было замерено с помощью функции chrono::high\_resolution\_clock::now(); из библиотеки <chrono> \cite{pythonlangtime}.

\section{Сведения о модулях программы}
Программа состоит из трех модулей:
\begin{enumerate}
	\item main.cpp - главный файл программы, в котором располагается меню;
	\item matr.cpp - файл программы, в котором располагаются все варианты умножения;
	\item iog.cpp - файл с вводом, выводом и генерацией матриц.
\end{enumerate}


\section{Реализация алгоритмов}
В листингах \ref{lst:quick_sort}, \ref{lst:shaker_sort}, \ref{lst:count_sort} представлены реализации алгоритмов умножения разреженных матриц (последовательный и параллельный).

\begin{lstlisting}[label=lst:quick_sort,caption=Реализация последовательного алгоритма умножения разреженных матриц]
matr *mutl(matr *a, matr *b, const int m)
{
	matr *new_matr = new(matr); // 1
	new_matr->rows_index.push_back(0);
	int col = 0;
	for (int i = 1; i < int(a->rows_index.size()); i++) // 2
	{
		int start = a->rows_index[i-1];
		int end = a->rows_index[i];
		int *prom = (int *)calloc(m, sizeof(int));
		for (int j = start; j < end; j++) { // 3
			int x = a->elements[j];
			for (int k = 0; k < int(b->elements.size()); k++) {
				int row;
				for (row = 0; row < int(b->rows_index.size() - 1); ++row) {
					if (b->rows_index[row] <= k && b->rows_index[row + 1] > k)
					break;
				}
				if (row == a->columns[j])
				prom[b->columns[k]] += x * b->elements[k]; // 3.a
			}
		}
		for (int f = 0; f < m; f++) // 3.b
		if (prom[f]) {
			new_matr->elements.push_back(prom[f]);
			new_matr->columns.push_back(f);
			col++;
		}
		new_matr->rows_index.push_back(col); // 3.c
		free(prom);
	}
	return new_matr; // 4
}
\end{lstlisting}

\begin{lstlisting}[label=lst:shaker_sort,caption=Реализация паралельного алгоритма умножения разреженных матриц]
void parallel_func(int **res, int st, int en, matr *a, matr *b, const int q)
{
	for (int i = st; i < en; i++) // 2
	{
		int start = a->rows_index[i-1];
		int end = a->rows_index[i];
		int *prom = (int *)calloc(q, sizeof(int));
		for (int j = start; j < end; j++) { // 3
			int x = a->elements[j];
			
			for (int k = 0; k < int(b->elements.size()); k++) {
				int row;
				for (row = 0; row < int(b->rows_index.size() - 1); ++row) {
					if (b->rows_index[row] <= k && b->rows_index[row + 1] > k)
					break;
				}
				if (row == a->columns[j])
				prom[b->columns[k]] += x * b->elements[k]; // 3.a
			}
		}
		res[i - 1] = prom;
	}
}

int **mutl_parallel(matr *a, matr *b, const int n, const int m, const int q, int thread_count)
{
	int **new_matr = (int**)calloc(n, sizeof(int*));
	vector<thread> threads(thread_count);
	double dc = double(n) / thread_count;
	int nows = 1;
	cout << thread_count << '\n';
	auto start = chrono::high_resolution_clock::now();
	for (auto& thr: threads) {
		thr = thread(parallel_func, new_matr, nows, int(round(nows + dc)), a, b, q);
		nows += dc;
	}
	auto stop = chrono::high_resolution_clock::now();
	auto dur1 = chrono::duration_cast<chrono::microseconds>((stop - start));
	cout << "Zapusk: " << dur1.count() << '\n';
	
	auto start2 = chrono::high_resolution_clock::now();
	for (auto& thr: threads)
	thr.join();
	auto stop2 = chrono::high_resolution_clock::now();
	auto dur2 = chrono::duration_cast<chrono::microseconds>((stop2 - start2));
	cout << "Simhro: " << dur2.count() << '\n';
	return new_matr; // 4
}
\end{lstlisting}

\section{Функциональные тесты}

В таблице \ref{tbl:functional_test} приведены тесты для функций, реализующих алгоритмы сортировки. Тесты пройдены успешно.

\begin{table}
	\begin{center}
		\captionsetup{singlelinecheck = false, justification=centering}
		\caption{Тестовые случаи}
		\begin{tabular}{l|l|l|l}
			N & Матрица 1 & Матрица 2 & Результат \\ 
			\hline
			\vspace{2mm}
			\vspace{2mm}
			1 & $\begin{pmatrix}
				1 & 1 & 1\\
				5 & 5 & 5\\
				2 & 2 & 2
			\end{pmatrix}$ &
			$\begin{pmatrix}
				1 \\
				1 \\
				1 
			\end{pmatrix}$ &
			$\begin{pmatrix}
				3 \\
				15 \\
				6
			\end{pmatrix}$ \\
			\vspace{2mm}
			\vspace{2mm}
			2 & $\begin{pmatrix}
				1 & 1 & 1
			\end{pmatrix}$ &
			$\begin{pmatrix}
				1 \\
				1 \\
				1
			\end{pmatrix}$ &
			$\begin{pmatrix}
				1 & 1 & 1\\
				1 & 1 & 1 \\
				1 & 1 & 1
			\end{pmatrix}$ \\
			\vspace{2mm}
			\vspace{2mm}
			3 & $\begin{pmatrix}
				1 & 1 \\
				1 & 1
			\end{pmatrix}$ &
			$\begin{pmatrix}
				1 & 1\\
				1 & 1
			\end{pmatrix}$ &
			$\begin{pmatrix}
				2 & 2\\
				2 & 2
			\end{pmatrix}$ \\
			\vspace{2mm}
			\vspace{2mm}
			4 & $\begin{pmatrix}
				2
			\end{pmatrix}$ &
			$\begin{pmatrix}
				2
			\end{pmatrix}$ &
			$\begin{pmatrix}
				4
			\end{pmatrix}$ \\
			\vspace{2mm}
			\vspace{2mm}
			5 & $\begin{pmatrix}
				1 & -2 & 3\\
				1 & 2 & 3\\
				1 & 2 & 3
			\end{pmatrix}$ &
			$\begin{pmatrix}
				-1 & 2 & 3\\
				1 & 2 & 3\\
				1 & 2 & 3
			\end{pmatrix}$ &
			$\begin{pmatrix}
				0 & 4 & 6\\
				4 & 12 & 18\\
				4 & 12 & 18
			\end{pmatrix}$\\
			\vspace{2mm}
			\vspace{2mm}
			6 & $\begin{pmatrix}
				1 & 2
			\end{pmatrix}$ &
			$\begin{pmatrix}
				1 & 2
			\end{pmatrix}$ &
			Некорректный размер\\
		\end{tabular}
	\end{center}
\end{table}

\FloatBarrier


\section*{Вывод}

Были реализованы алгоритмы умножения разреженных матриц: последовательный и параллельный.
