\chapter{Технологическая часть}
В данном разделе будут приведены средства реализации и листинги реализованных алгоритмов.

\section{Средства реализации}
В данной работе для реализации был выбран язык программирования $Python$ \cite{pythonlang}. В текущей лабораторной работе требуется замерить процессорное время работы выполняемой программы
и визуализировать результаты при помощи графиков. Инструменты для этого присутствуют в выбранном языке программирования.

\section{Сведения о модулях программы}
Программа состоит из одного модулей: $main.py$ --- файл, содержащий все.

\section{Функциональное тестирование}

В данном разделе будет приведена таблица с тестами (таблица \ref{table:ref1}).
\begin{center}
	\captionsetup{justification=raggedright,singlelinecheck=off}
	\begin{table}[ht]
		\centering
		\caption{Таблица тестов}
		\label{table:ref1}
		\begin{tabular}{ |c|c|c|}
			\hline
			Входные данные    & Пояснение   	  & Результат    \\ 
			\hline
			пара норок			  & Первый элемент   & Ответ верный \\ \hline
			куча 			  & Средний элемент    & Ответ верный \\ \hline
			гора ужаса 		  & Последний элемент & Ответ верный \\ \hline
			упс & Несуществующий элемент & Ответ верный (-1) \\ \hline
			123 & Несуществующий элемент & Ответ верный (-1) \\ \hline
		\end{tabular}
	\end{table}
\end{center}
Все тесты пройдены успешно.


\section{Вывод}
В данном разделе был представлен листинг рассматриваемого алгоритма поиска в словаре, приведена информация о средствах реализации, сведения о модулях программы и было проведено функциональное тестирование.