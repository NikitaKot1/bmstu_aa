\chapter{Исследовательская часть}

В данном разделе будут приведены примеры работы программ, постановка эксперимента и сравнительный анализ алгоритмов на основе полученных данных.

\section{Технические характеристики}

Технические характеристики устройства, на котором выполнялось тестирование:

\begin{itemize}
	\item операционная система: Manjaro xfce \cite{ubuntu} Linux \cite{linux} x86\_64;
	\item память --- 8 Гб;
	\item мобильный процессор AMD Ryzen™ 7 3700U @ 2.3 ГГц \cite{intel}.
\end{itemize}

Тестирование проводилось на ноутбуке, включенном в сеть электропитания. Во время тестирования ноутбук был нагружен только встроенными приложениями окружения, а также непосредственно системой тестирования.

\section{Демонстрация работы программы}

На рисунке \ref{img:primer} представлен результат работы программы.

\img{32mm}{primer}{Пример работы программы}
\FloatBarrier

\section{Формализация объекта и его признака}
\label{formal}
Согласно согласованному варианту, формализуем объект <<муравейник>> следующим образом: определим набор данных и признак объекта, на основании которого составим набор термов. Согласно варианту, признаком, по которому будет производиться поиск объектов, будет являться \textit{размер} в относительных величинах восприятия --- целое число.

Определим следующие термы, соответствующие признаку <<размер>>:
\begin{enumerate}
	\item <<Пара норок>>;
	\item <<Земляное решето>>;
	\item <<Куча>>;
	\item <<Пара кучек>>;
	\item <<Большая куча>>;
	\item <<Гора ужаса>>.
\end{enumerate}

\subsection{Построение функции принадлежности термам}

Построим графики функций принадлежности числовых значений переменной термам, описывающим группы значений лингвистической переменной.

Для этого для каждого значения из размера муравейника для каждого терма из перечисленных найдём количество респондентов, согласно которым значение удовлетворяет сопоставляемому терму.
Данное значение поделим на количество респондентов --- это и будет значением функции $\mu$ для терма в точке.
Графики функций принадлежности числовых значений роста термам, приведён на рисунке \ref{img:fig1}.

\img{100mm}{fig1}{Графики функций принадлежности числовых значений переменной термам, описывающим группы значений лингвистической переменной}
