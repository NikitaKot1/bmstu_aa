\chapter{Технологическая часть}

В данном разделе будут приведены требования к программному обеспечению, средства реализации и листинги кода.

\section{Требования к ПО}

К программе предъявляется ряд требований:
\begin{itemize}
	\item ПО принимает две целочисленные матрицы размерностью не больше 700;
	\item ПО возвращает результирующую матрицу --- произведение двух входных матриц.
\end{itemize}


\section{Средства реализации}

В качестве языка программирования для реализации данной лабораторной работы был выбран язык программирования Си \cite{pythonlang}. 

В данном языке есть все требующиеся для данной лабораторной инструменты разработки. 

Процессорное время работы реализаций алгоритмов было замерено с помощью функции tick() из библиотеки sys/time.h \cite{pythonlangtime}.

\section{Сведения о модулях программы}
Программа состоит из четырех модулей:
\begin{enumerate}[label=\arabic*)]
	\item main.c --- главный файл программы, в котором вызываются остальные файлы;
	\item interface.c --- файл программы, в котором располагается меню;
	\item algos.c --- файл программы, в котором располагаются коды всех алгоритмов;
	\item data.c --- файл работой с памятью.
\end{enumerate}


\section{Реализация алгоритмов}
В листингах \ref{lst:stand}, \ref{lst:vin}, \ref{lst:vinopt} представлены реализации алгоритмов умножения матриц: стандартный, Винограда, оптимизированный Винограда.

\begin{lstlisting}[label=lst:stand,caption=Реализация стандартного алгоритма умножения матриц]
int **multStand(int ** m1, int ** m2, int l, int m, int n)
{
	int **res = allocateMatrix(l, n);
	if (res) {
		for (int i = 0; i < l; i++) {
			
			for (int j = 0; j < n; j++) {
				res[i][j] = 0;
				for (int r = 0; r < m; r++)
				res[i][j] = res[i][j] + m1[i][r] * m2[r][j];
			}
		}
	}
	else
	printf("Error !\n");
	return res;
}
\end{lstlisting}

\begin{lstlisting}[label=lst:vin,caption=Реализация алгоритма умножения матриц Винограда]
int **multVin(int **m1, int **m2, int l, int m, int n)
{
	int **res = allocateMatrix(l, n);
	int *mulH = calloc (l, sizeof(int));
	int *mulV = calloc (n, sizeof(int));
	if (res && mulH && mulV) {
		for (int i = 0; i < l; i++)
		for (int k = 0; k < m / 2; k++)
		mulH[i] = mulH[i] + m1[i][2 * k] * m1[i][2 * k + 1];
		for (int i = 0; i < n; i++)
		for (int k = 0; k < m / 2; k++)
		mulV[i] = mulV[i] + m2[2 * k][i] * m2[2 * k + 1][i];
		for (int i = 0; i < l; i++)
		for (int j = 0; j < n; j++) {
			res[i][j] = - mulH[i] - mulV[j];
			for (int k = 0; k < m / 2; k++)
			res[i][j] = res[i][j] + (m1[i][2 * k] + m2[2 * k + 1][j]) * (m1[i][2 * k + 1] + m2[2 * k][j]);
		}
		if (m % 2 == 1)
		for (int i = 0; i < l ; i++)
		for (int j = 0; j < n ; j++)
		res[i][j] = res[i][j] + m1[i][m - 1] * m2 [m - 1][j];
		free(mulH);
		free(mulV);
	}
	else
	printf("Error!\n");
	return res;
}
\end{lstlisting}

\begin{lstlisting}[label=lst:vinopt,caption=Реализация оптимизированного алгоритма умножения матриц Винограда ]
int **multVinOptimize(int **m1, int **m2, int l, int m, int n)
{
	int **res = allocateMatrix(l, n);
	int *mulH = calloc(l, sizeof(int));
	int *mulV = calloc(n, sizeof(int));
	if (res && mulH && mulV) {
		int optm = m / 2;
		int f;
		for (int i = 0; i < l; i++)
		for (int k = 0; k < optm; k++) {
			f = k << 1;
			mulH[i] += m1[i][f] * m1[i][f + 1];
		}
		for (int i = 0; i < n; i++)
		for (int k = 0; k < optm; k++) {
			f = k << 1;
			mulV[i] += m2[f][i] * m2[f + 1][i];
		}
		for (int i = 0; i < l; i++)
		for (int j = 0; j < n; j++) {
			res[i][j] = - mulH[i] - mulV[j];
			for (int k = 0; k < optm; k++) {
				f = k << 1;
				res[i][j] += (m1[i][f] + m2[f+ 1][j]) * (m1[i][f + 1] + m2[f][j]);
			}
		}
		if (m % 2 == 1)
		for (int i = 0; i < l ; i++)
		for (int j = 0; j < n ; j++)
		res[i][j] += m1[i][m - 1] * m2 [m - 1][j];
		free(mulH);
		free(mulV);
	}
	else
	printf("Error!\n");
	return res;
}
\end{lstlisting}

\section{Функциональные тесты}

В таблице 3.1 приведены тестовые случаи для алгоритмов умножения матриц. Случаи 1--2 --- умножение однострочных/одностолбцовых матриц, 3--5 --- квадратных матриц одного размера, 6 --- тест на некорректный размер.

\begin{table}
	\begin{center}
		\captionsetup{singlelinecheck = false, justification=centering}
		\caption{Тестовые случаи}
		\begin{tabular}{l|l|l|l}
			N & Матрица 1 & Матрица 2 & Результат \\ 
			\hline
			\vspace{2mm}
			\vspace{2mm}
			1 & $\begin{pmatrix}
				1 & 1 & 1\\
				5 & 5 & 5\\
				2 & 2 & 2
			\end{pmatrix}$ &
			$\begin{pmatrix}
				1 \\
				1 \\
				1 
			\end{pmatrix}$ &
			$\begin{pmatrix}
				3 \\
				15 \\
				6
			\end{pmatrix}$ \\
			\vspace{2mm}
			\vspace{2mm}
			2 & $\begin{pmatrix}
				1 & 1 & 1
			\end{pmatrix}$ &
			$\begin{pmatrix}
				1 \\
				1 \\
				1
			\end{pmatrix}$ &
			$\begin{pmatrix}
				1 & 1 & 1\\
				1 & 1 & 1 \\
				1 & 1 & 1
			\end{pmatrix}$ \\
			\vspace{2mm}
			\vspace{2mm}
			3 & $\begin{pmatrix}
				1 & 1 \\
				1 & 1
			\end{pmatrix}$ &
			$\begin{pmatrix}
				1 & 1\\
				1 & 1
			\end{pmatrix}$ &
			$\begin{pmatrix}
				2 & 2\\
				2 & 2
			\end{pmatrix}$ \\
			\vspace{2mm}
			\vspace{2mm}
			4 & $\begin{pmatrix}
				2
			\end{pmatrix}$ &
			$\begin{pmatrix}
				2
			\end{pmatrix}$ &
			$\begin{pmatrix}
				4
			\end{pmatrix}$ \\
			\vspace{2mm}
			\vspace{2mm}
			5 & $\begin{pmatrix}
				1 & -2 & 3\\
				1 & 2 & 3\\
				1 & 2 & 3
			\end{pmatrix}$ &
			$\begin{pmatrix}
				-1 & 2 & 3\\
				1 & 2 & 3\\
				1 & 2 & 3
			\end{pmatrix}$ &
			$\begin{pmatrix}
				0 & 4 & 6\\
				4 & 12 & 18\\
				4 & 12 & 18
			\end{pmatrix}$\\
			\vspace{2mm}
			\vspace{2mm}
			6 & $\begin{pmatrix}
				1 & 2
			\end{pmatrix}$ &
			$\begin{pmatrix}
				1 & 2
			\end{pmatrix}$ &
			Некорректный размер\\
		\end{tabular}
	\end{center}
\end{table}

\FloatBarrier

\section*{Вывод}


На основе схем из конструкторского раздела были написаны реализации требуемых алгоритмов, а также проведено их тестирование.