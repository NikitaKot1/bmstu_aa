\chapter{Исследовательская часть}

В данном разделе будут приведены примеры работы программ, постановка эксперимента и сравнительный анализ алгоритмов на основе полученных данных.

\section{Технические характеристики}

Технические характеристики устройства, на котором выполнялось тестирование:

\begin{itemize}
	\item операционная система Manjaro xfce \cite{ubuntu} Linux \cite{linux} x86\_64;
	\item память 8 ГБ;
	\item мобильный процессор AMD Ryzen™ 7 3700U @ 2.3 ГГц \cite{intel}.
\end{itemize}

Тестирование проводилось на ноутбуке, включенном в сеть электропитания. Во время тестирования ноутбук был нагружен только встроенными приложениями окружения, а также непосредственно системой тестирования.

\section{Демонстрация работы программы}

На рисунке \ref{img:primer} представлен результат работы программы: в демонстрации был выбран вариант перемножения случайных матриц корректного размера.

\img{120mm}{primer}{Пример работы программы}

\FloatBarrier

\section{Время выполнения реализаций алгоритмов}

В таблицах 4.1--4.3 представлены замеры процессорного времени работы алгоритмов на квадратных матрицах размером от 50 до 550 элементов. Такая размерность выбрана, потому что на больших размерах массива происходит переполнение типа счетчика тиков. Каждое значение было получено усреднением по 10 замерам. Примем за лучший случай четную размерность $N$, за худший --- нечетную. Время приведено в тиках процессора.

\begin{table}
	\centering
	\caption{Замеры времени работы реализаций алгоритмов для лучшего случая}
	\begin{tabular}{l|r|r|r|r|r|r|r|r|r}
		\text{N} & \text{Стандартный} & \text{Винограда} & \text{оптимизированный Винограда}\\
		\hline
		50 & 1000000 & 1000000 & 1000000\\
		100 & 15000000 & 12000000 & 10000000\\
		150 & 63000000 & 52000000 & 40000000\\
		200 & 185000000 & 152000000 & 120000000\\
		250 & 418000000 & 341000000 & 270000000\\
		300 & 844000000 & 693000000 & 545000000\\
		350 & 1572000000 & 1297000000 & 1023000000\\
		400 & 2630000000 & 2168000000 & 1709000000\\
		450 & 4187000000 & 3449000000 & 2723000000\\
		500 & 6434000000 & 5462000000 & 4481000000\\
		550 & 9540000000 & 7948000000 & 6432000000\\
	\end{tabular}
\end{table}

\FloatBarrier

\begin{table}
	\centering
	\caption{Замеры времени работы реализаций алгоритмов для худшего случая}
	\begin{tabular}{l|r|r|r|r|r|r|r|r|r}
		\text{N} & \text{Стандартный} & \text{Винограда} & \text{оптимизированный Винограда}\\
		\hline
		51 & 1000000 & 1000000 & 1000000 \\
		101 & 18000000 & 12000000 & 9000000\\
		151 & 66000000 & 51000000 & 40000000\\
		201 & 186000000 & 149000000 & 117000000\\
		251 & 443000000 & 358000000 & 283000000\\
		301 & 864000000 & 707000000 & 557000000\\
		351 & 1563000000 & 1282000000 & 1011000000\\
		401 & 2678000000 & 2188000000 & 1717000000\\
		451 & 4491000000 & 3719000000 & 2912000000\\
		501 & 6669000000 & 5498000000 & 4318000000\\
		551 & 9705000000 & 7972000000 & 6261000000\\
	\end{tabular}
\end{table}

\FloatBarrier

На рисунке \ref{time_best} представлены зависимости времени работы различных реализаций алгоритмов от линейного размера квадратных матриц с четной размерностью. В результате эксперимента было получено, что на квадратных матрицах размером до 550 оптимизированный алгоритм Винограда работает на 40 \% быстрее стандартного и на 25 \% быстрее классического алгоритма Винограда.

\begin{figure}
	\captionsetup{singlelinecheck = false, justification=centering}
	\centering
	\begin{tikzpicture}
		\begin{axis}[
			xlabel={размер массива},
			ylabel={время, тики},
			width=0.95\textwidth,
			height=0.3\textheight,
			xmin=0, xmax=600,
			legend pos=north west,
			xmajorgrids=true,
			grid style=dashed,
			]
			\addplot[
			color=blue,
			mark=asterisk
			]
			table [x=N, y=time]{
				N time
				50 1000000
				100 15000000
				150 63000000
				200 185000000
				250 418000000
				300 844000000
				350 1572000000
				400 2630000000
				450 4187000000
				500 6434000000
				550 9540000000
			};
			\addplot[
			color=red,
			mark=asterisk
			]
			table [x=N, y=time]{
				N time
				50 1000000
				100 12000000
				150 52000000
				200 152000000
				250 341000000
				300 693000000
				350 1297000000
				400 2168000000
				450 3449000000
				500 5462000000
				550 7948000000
			};
			\addplot[
			color=yellow,
			mark=asterisk
			]
			table [x=N, y=time]{
				N time
				50 1000000
				100 10000000
				150 40000000
				200 120000000
				250 270000000
				300 545000000
				350 1023000000
				400 1709000000
				450 2723000000
				500 4481000000
				550 6432000000
			};
			\legend{Стандартный, Винограда, Винограда (оптим.)}
		\end{axis}
	
	\end{tikzpicture}
	\caption{Сравнение времени работы реализаций алгоритмов для лучшего случая}
	\label{time_best}
\end{figure}

\FloatBarrier

На рисунке \ref{time_worst} представлены зависимости времени работы различных реализаций алгоритмов от линейного размера квадратных матриц с нечетной размерностью. В результате эксперимента было получено, что на квадратных матрицах размером до 551 реализации алгоритмов показывают примерно такие же результаты.

\begin{figure}
	\captionsetup{singlelinecheck = false, justification=centering}
	\centering
	\begin{tikzpicture}
		\begin{axis}[
			xlabel={размер массива},
			ylabel={время, тики},
			width=0.95\textwidth,
			height=0.3\textheight,
			xmin=0, xmax=601,
			legend pos=north west,
			xmajorgrids=true,
			grid style=dashed,
			]
			\addplot[
			color=blue,
			mark=asterisk
			]
			table [x=N, y=time]{
				N time
				51 1000000
				101 18000000
				151 66000000
				201 186000000
				251 443000000
				301 864000000
				351 1563000000
				401 2678000000
				451 4491000000
				501 6669000000
				551 9705000000
			};
			\addplot[
			color=red,
			mark=asterisk
			]
			table [x=N, y=time]{
				N time
				51 1000000
				101 12000000
				151 51000000
				201 149000000
				251 358000000
				301 707000000
				351 1282000000
				401 2188000000
				451 3719000000
				501 5498000000
				551 7972000000
			};
			\addplot[
			color=yellow,
			mark=asterisk
			]
			table [x=N, y=time]{
				N time
				51 1000000
				101 9000000
				151 40000000
				201 117000000
				251 283000000
				301 557000000
				351 1011000000
				401 1717000000
				451 2912000000
				501 4318000000
				551 6261000000
			};
			\legend{Стандартный, Винограда, Винограда (оптим.)}
		\end{axis}
	\end{tikzpicture}
	\caption{Сравнение времени работы реализаций алгоритмов для худшего случая}
	\label{time_worst}
\end{figure}

\FloatBarrier

\section*{Вывод}

Получены следуюшие результаты для реализации стандартного алгоритма умножения матриц:
\begin{itemize}
	\item в лучшем случае она на 15 \% медленнее реализации алгоритма Винограда и на 30 \% медленнее реализации оптимизированного алгоритма Винограда;
	\item в худшем случае она на 18 \% медленнее реализации алгоритма Винограда и на 35 \% медленнее реализации оптимизированного алгоритма Винограда.
\end{itemize}

Также оптимизированный алгоритм Винограда дает выигрыш по затрачиваемому времени на 20 \% относительно стандартного алгоритма Винограда независимо от класса эквивалентности данных.

