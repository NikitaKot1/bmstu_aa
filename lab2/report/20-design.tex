\chapter{Конструкторская часть}
В этом разделе будут приведены схемы алгоритмов и вычисления трудоемкости данных алгоритмов.

\section{Разработка алгоритмов}

На рисунках  \ref{img:stand}, \ref{img:vino} и \ref{img:optvino} представлены схемы алгоритмов стандартного умножения матриц, умножения Винограда и оптимизированного умножения Винограда.


\img{160mm}{stand}{Схема алгоритма стандартного умножения матриц}

\FloatBarrier

\img{240mm}{vino}{Схема алгоритма умножения матриц Винограда}

\FloatBarrier

\img{190mm}{optvino}{Схема оптимизированного алгоритма умножения матриц Винограда}

\FloatBarrier

\section{Модель вычислений (оценки трудоемкости)}

Для последующего вычисления трудоемкости необходимо ввести модель вычислений.
\begin{enumerate}
	\item Операции из списка (\ref{for:opers2}) имеют трудоемкость 1:
	\begin{equation}
		\label{for:opers2}
		+, ++, +=, -, {-}-, -=, !=, <, >, <=, >=, <<, >>, []
	\end{equation}
	\item Операции из списка (\ref{for:opers1}) имеют трудоемкость 2:
	\begin{equation}
		\label{for:opers1}
		/, /=, *, *= ,\%, \%=
	\end{equation}
	\item Трудоемкость оператора выбора \code{if условие then A else B} рассчитывается как
	\begin{equation}
		\label{for:if}
		f_{if} = f_{\text{условия}} +
		\begin{cases}
			f_A, & \text{если условие выполняется,}\\
			f_B, & \text{иначе.}
		\end{cases}
	\end{equation}
	\item Трудоемкость цикла с $N$ итерациями рассчитывается как
	\begin{equation}
		\label{for:for}
		f_{for} = f_{\text{инициализации}} + f_{\text{сравнения}} + N(f_{\text{тела}} + f_{\text{инкремент}} + f_{\text{сравнения}})
	\end{equation}
	\item Трудоемкость вызова функции равна 0.
\end{enumerate}

\section{Трудоёмкость алгоритмов}

\subsection{Трудоемкость стандартного алгоритма}
Трудоемкость стандартного алгоритма состоит из следующих составляющих:

\begin{itemize}
	\item трудоемкость цикла по i $\in$ [1, L] $f = 2 + L(2 + f_{body})$;
	\item трудоемкость цикла по j $\in$ [1, N] $f = 2 + 3 + N(2 + f_{body})$;
	\item трудоемкость цикла по r $\in$ [1, M] $f = 2 + M(2 + 12) = 2 + 14M$.
\end{itemize}

Итого, трудоемкость стандартного алгоритма равна

	$ f = 2 + L(4 + N(7 + 14M)) = 2 + 4L + 7LN + 14LNM \approx 14LNM \approx O(N^3) $

\subsection{Трудоемкость алгоритма Винограда}

Трудоемкость алгоритма Винограда состоит из следующих составляющих:
\begin{itemize}
	\item аллокация и инициализация векторов mulH и mulV --- $f_{HV} = L + N + 2 + L(2 + 3 + \frac{1}{2}M(3 + 14)) + 2 + N(2 + 3 + \frac{1}{2}M(3 + 14)) = 2 + 6L + 6N + \frac{17}{2}LM + \frac{17}{2}NM$;
	\item цикл по i $\in$ [1, L] $f_{L} = 2 + L(2 + f_{body})$;
	\item цикл по j $\in$ [1, N] $f_{N} = 2 + N(2 + 7 + f_{body})$;
	\item цикл по k $\in$ [1, M/2] $f_{M/2} = 3 + \frac{1}{2}M(3 + 28) = 3 + \frac{31}{2}M$;
	\item проверка размеров на нечетность
	\begin{equation}
		f_{if} = 3 + 
		\begin{cases}
			f_A, & \text{если условие выполняется,}\\
			f_B, & \text{иначе.}
		\end{cases}
	\end{equation}
\end{itemize}

Итого, трудоемкость алгоритма Винограда для лучшего случая, который наступает при четной размерности $M$ умножаемых матриц, равна

$ f = 4 + 10L + 6N + \frac{17}{2}LM + \frac{17}{2}NM + 12LN + \frac{31}{2}LNM \approx 15,5LNM \approx O(N^3) $

Для худшего случая, который наступает при нечетной размерности $M$ умножаемых матриц, равна

$f = 6 + 14L + 6N + \frac{17}{2}LM + \frac{17}{2}NM + 25LN + \frac{31}{2}LNM \approx 15,5LNM \approx O(N^3) $

\subsection{Трудоемкость оптимизированного алгоритма Винограда}

Трудоемкость оптимизированного алгоритма Винограда состоит из следующих составляющих:
\begin{itemize}
	\item аллокация и инициализация векторов mulH и mulV --- $f_{HV} = L + N + 2 + L(2 + 3 + \frac{1}{2}M(3 + 13)) + 2 + N(2 + 3 + \frac{1}{2}M(3 + 13)) = 2 + 6L + 6N + 8LM + 8NM$;
	\item цикл по i $\in$ [1, L] $f_{L} = 2 + L(2 + f_{body})$;
	\item цикл по j $\in$ [1, N] $f_{N} = 2 + N(2 + 5 + f_{body} + f_{if} + 3)$;
	\item цикл по k $\in$ [1, M/2] $f_{M/2} = 3 + \frac{1}{2}M(3 + 23) = 3 + 13M$;
	\item проверка размеров на нечетность
	\begin{equation}
		f_{if} = 3 + 
		\begin{cases}
			f_A, & \text{если условие выполняется,}\\
			f_B, & \text{иначе.}
		\end{cases}
	\end{equation}
\end{itemize} 



\section*{Вывод}

Были разработаны схемы всех трех алгоритмов умножения матриц. Для каждого из них были рассчитаны и оценены лучшие и худшие случаи.


