\chapter{Технологическая часть}

В данном разделе будут приведены требования к программному обеспечению, средства реализации и листинги кода.

\section{Требования к программному обеспечению}

К программе предъявляется ряд требований:

\begin{itemize}[label=--]
	\item на вход подается имя файла с матрицей смежности городов;
	\item программа определяет количество дней и коэффициенты;
	\item возможно измерение реального времени;
\end{itemize}

\section{Средства реализации}

В качестве языка программирования для реализации данной лабораторной работы был выбран язык программирования Python \cite{pythonlang}. 

В данном языке есть все требующиеся для данной лабораторной инструменты разработки. 

Время работы реализаций алгоритмов было замерено с помощью функции process\_time() из библиотеки time \cite{pythonlangtime}

\section{Реализация алгоритмов}
В листинге \ref{lst:quick_sort} представлена реализация алгоритма полного перебора путей.
\\
\\

\begin{lstlisting}[label=lst:quick_sort,caption=Реализация последовательного алгоритма умножения разреженных матриц]
def fullCombinationAlg(matrix, size):
	places = np.arange(size)
	placesCombinations = list()

	for combination in it.permutations(places):
		combArr = list(combination)
		placesCombinations.append(combArr)

	minDist = float("inf")

	for i in range(len(placesCombinations)):
		placesCombinations[i].append(placesCombinations[i][0])
		curDist = 0
		for j in range(size):
			startCity = placesCombinations[i][j]
			endCity = placesCombinations[i][j + 1]
			curDist += matrix[startCity][endCity]

		if (curDist < minDist):
			minDist = curDist
			bestWay = placesCombinations[i]

	return minDist, bestWay
\end{lstlisting}

В листинге \ref{lst:shaker_sort} представлена реализация муравьиного алгоритма поиска путей.

\begin{lstlisting}[label=lst:shaker_sort,caption=Реализация муравьиного алгоритма поиска путей]
def calcQ(matrix, size):
	q = 0
	count = 0
	for i in range(size):
		for j in range(size):
			if (i != j):
				q += matrix[i][j]
				count += 1
	return q / count

def calcPheromones(size):
	min_phero = 1
	pheromones = [[min_phero for i in range(size)] for j in range(size)]
	return pheromones


def calcVisibility(matrix, size):
	visibility = [[(1.0 / matrix[i][j] if (i != j) else 0) for j in range(size)] for i in range(size)]
	return visibility


def calcVisitedPlaces(route, ants):
	visited = [list() for _ in range(ants)]
	for ant in range(ants):
		visited[ant].append(route[ant])
	return visited


def calcLength(matrix, route):
	length = 0
	for way_len in range(1, len(route)):
		length += matrix[route[way_len - 1], route[way_len]]
	return length


def updatePheromones(matrix, places, visited, pheromones, q, k_evaporation):
	ants = places
	for i in range(places):
		for j in range(places):
			delta = 0
			for ant in range(ants):
				length = calcLength(matrix, visited[ant])
				delta += q / length
			pheromones[i][j] *= (1 - k_evaporation)
			pheromones[i][j] += delta
			if (pheromones[i][j] < MIN_PHEROMONE):
				pheromones[i][j] = MIN_PHEROMONE
	return pheromones


def findWays(pheromones, visibility, visited, places, ant, alpha, beta):
	pk = [0] * places
	for place in range(places):
		if place not in visited[ant]:
			ant_place = visited[ant][-1]
			pk[place] = pow(pheromones[ant_place][place], alpha) * \
				pow(visibility[ant_place][place], beta)
		else:
			pk[place] = 0
	sum_pk = sum(pk)
	for place in range(places):
		pk[place] /= sum_pk
	return pk


def chooseNextPlaceByPosibility(pk):
	posibility = random()
	choice = 0
	chosenPlace = 0
	while ((choice < posibility) and (chosenPlace < len(pk))):
		choice += pk[chosenPlace]
		chosenPlace += 1
	return chosenPlace



def antAlgorithm(matrix, places, alpha, beta, k_evaporation, days):
	q = calcQ(matrix, places)
	bestWay = []
	minDist = float("inf")
	pheromones = calcPheromones(places)
	visibility = calcVisibility(matrix, places)
	ants = places
	for day in range(days):
		route = np.arange(places)
		visited = calcVisitedPlaces(route, ants)
		for ant in range(ants):
			while (len(visited[ant]) != ants):
				pk = findWays(pheromones, visibility, visited, places, ant, alpha, beta)
				chosenPlace = chooseNextPlaceByPosibility(pk)
				visited[ant].append(chosenPlace - 1)
			visited[ant].append(visited[ant][0])
			curLength = calcLength(matrix, visited[ant])
			if (curLength < minDist):
				minDist = curLength
				bestWay = visited[ant]
		pheromones = updatePheromones(matrix, places, visited, pheromones, q, k_evaporation)
	return minDist, bestWay
\end{lstlisting}

\FloatBarrier

\section{Функциональные тесты}

В таблице \ref{tbl:functional_test} приведены тесты для функций, реализующих алгоритмы сортировки.

\begin{center}
	\captionsetup{justification=raggedright,singlelinecheck=off}
	\begin{longtable}[c]{|c|c|c|c|c|}
		\caption{Функциональные тесты\label{tbl:functional_test}} \\ \hline
		Матрица смежности & Ожидаемый результат & Результат программы \\
		\hline		
		$ \begin{pmatrix}
			0 & 1 & 2 \\
			1 & 0 & 1 \\
			2 & 1 & 0	
		\end{pmatrix}$ &
		4, [0, 1, 2, 0] &
		4, [0, 1, 2, 0] \\
		
		$ \begin{pmatrix}
			0 & 15 & 19 & 20 \\
			15 &  0 & 12 & 13 \\
			19 & 12 &  0 & 17 \\
			20 & 13 & 17 &  0
		\end{pmatrix}$ &
		64, [0, 1, 2, 3, 0] &
		64, [0, 1, 2, 3, 0] \\
		
		$ \begin{pmatrix}
			0 &  4 &  2 &  1 & 7 \\
			4 &  0 &  3 &  7 & 2 \\
			2 &  3 &  0 & 10 & 3 \\
			1 &  7 & 10 &  0 & 9 \\
			7 &  2 &  3 &  9 & 0
		\end{pmatrix}$ &
		15, [0, 2, 4, 1, 3, 0] &
		15, [0, 2, 4, 1, 3, 0] \\
		\hline
	\end{longtable}
\end{center}

\FloatBarrier


\section*{Вывод}

Написано и протестировано программное обеспечение для поставленной задачи.
