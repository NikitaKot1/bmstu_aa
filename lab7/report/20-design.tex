\chapter{Конструкторская часть}
В этом разделе будут приведены схемы алгоритмов и описаны используемые типы данных.

\section{Описание используемых типов данных}

При реализации алгоритмов будут использоваться следующие типы данных:
\begin{itemize}[label=--]
	\item размер матрицы смежности --- целое число;
	\item имя файла --- строка;
	\item коэффициенты $\alpha, \beta$, \textit{k\_evaporation} --- действительные числа;
	\item матрица смежности --- матрица целых чисел.
\end{itemize}

\section{Разработка алгоритмов}

На рисунке \ref{img:perebor.pdf} представлена схема алгоритма полного перебора.

\img{185mm}{perebor.pdf}{Схема алгоритма полного перебора поиска путей}

\FloatBarrier

На рисунке \ref{img:angs.pdf} представлена схема муравьиного алгоритма.

\img{180mm}{angs.pdf}{Схема муравьиного алгоритма умножения поиска путей}

\FloatBarrier

\section*{Вывод}

Были разработаны схемы алгоритмов поиска путей (полный перебор и муравьиный).

