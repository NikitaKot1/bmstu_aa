\chapter{Конструкторская часть}
В этом разделе будут приведены схемы алгоритмов и вычисления трудоемкости данных алгоритмов.

%\section{Требования к вводу}

%К вводу предъявляются следующие требования.
%\begin{itemize}
%	\item на вход программа получает выбранный пункт меню (цифра от 0 до 6);
%	\item на вход подается массив из целых чисел.
%\end{itemize}

%\section{Требования к программе}

%К программе продъявляются следующие требования.
%\begin{enumerate}
%	\item Две пустые строки - корректный ввод, программа не должна аварийно завершаться.
%	\item На выход программа должна вывести число - расстояние Левенштейна (Дамерау-Левенштейна), матрицу при необходимости.
%\end{enumerate}

\section{Разработка алгоритмов}

На рисунках  \ref{img:peremesh}, \ref{img:quick} и \ref{img:count} представлены схемы алгоритмов сортировки пузырьком, выбором и вставками соответственно.


\img{170mm}{peremesh}{Схема алгоритма сортировки перемешиванием}
\img{240mm}{quick}{Схема алгоритма быстрой сортировки}
\img{170mm}{count}{Схема алгоритма сортировки подсчетом}

\section{Модель вычислений (оценки трудоемкости)}

Для последующего вычисления трудоемкости необходимо ввести модель вычислений.
\begin{enumerate}
	\item Операции из списка (\ref{for:opers2}) имеют трудоемкость 1:
	\begin{equation}
		\label{for:opers2}
		+, ++, +=, -, {-}-, -=, !=, <, >, <=, >=, <<, >>, []
	\end{equation}
	\item Операции из списка (\ref{for:opers1}) имеют трудоемкость 2:
	\begin{equation}
		\label{for:opers1}
		/, /=, *, *= ,\%, \%=
	\end{equation}
	\item трудоемкость оператора выбора \code{if условие then A else B} рассчитывается как
	\begin{equation}
		\label{for:if}
		f_{if} = f_{\text{условия}} +
		\begin{cases}
			f_A, & \text{если условие выполняется,}\\
			f_B, & \text{иначе.}
		\end{cases}
	\end{equation}
	\item трудоемкость цикла рассчитывается, как (\ref{for:for});
	\begin{equation}
		\label{for:for}
		f_{for} = f_{\text{инициализации}} + f_{\text{сравнения}} + N(f_{\text{тела}} + f_{\text{инкремент}} + f_{\text{сравнения}})
	\end{equation}
	\item трудоемкость вызова функции равна 0.
\end{enumerate}

\section{Трудоёмкость алгоритмов}

Обозначим во всех последующих вычислениях размер массивов как N.

\subsection{Алгоритм сортировки перемешиванием}


Трудоёмкость сравнения внешнего цикла $WHILE(swap == True)$, которая равна (\ref{for:bubble_outer}):
\begin{equation}
	\label{for:bubble_outer}
	f_{outer} = 1 + 2 \cdot (N - 1)
\end{equation}

Суммарная трудоёмкость внутренних циклов, количество итераций которых меняется в промежутке $[1..N-1]$, которая равна
\begin{equation}
	\label{for:bubble_inner}
	f_{inner} = 5(N - 1) + \frac{2 \cdot (N - 1)}{2} \cdot (3 + f_{if})
\end{equation}

Трудоёмкость условия во внутреннем цикле, которая равна
\begin{equation}
	\label{for:bubble_if}
	f_{if} = 4 + \begin{cases}
		0, & \text{л.с.}\\
		9, & \text{х.с.}\\
	\end{cases}
\end{equation}

Трудоёмкость в лучшем случае, который наступает при отсортированном массиве (\ref{for:bubble_best}):
\begin{equation}
	\label{for:bubble_best}
	f_{best} = -3 + \frac{3}{2} N + \approx \frac{3}{2} N = O(N)
\end{equation}

Трудоёмкость в худшем случае, который наступает при обратно отсортированном массиве (\ref{for:bubble_worst}):
\begin{equation}
	\label{for:bubble_worst}
	f_{worst} = -3 - 8N + 8N^2 \approx 8N^2 = O(N^2)
\end{equation}



\subsection{Алгоритм быстрой сортировки}


Трудоёмкость алгоритма быстрой сортировки состоит из следующих составляющих.
\begin{itemize}
	\item Трудоёмкость вычисления опорного элемента разбиения массива на 2 части (\ref{for:outer_quick}): трудоемкость сравнения в цикле, инкремента во внешнем цикле, а также зависимых только от цикла операций, по $i \in [1..m)$, где $m \in N/2^k$, где $k \in [0..f)$, где $f$ равна максимальной глубине стека рекурсии:
	\begin{equation}
		\label{for:outer_quick}
		f_{part} = 2 + 12(m - 1)
	\end{equation}
	\item Каждый рекурсивный вызов порождает 2 других рекурсивных вызова, кроме тривиальных случаев, таким образом максимальная глубина вызова стека будет равна $log_2(N)$ в общем случае.
	\item Трудоёмкость условия во внутреннем цикле, которая равна
	\begin{equation}
		\label{for:quick_if}
		f_{if} = 3 + \begin{cases}
			0, & \text{л.с.}\\
			8, & \text{х.с.}\\
		\end{cases}
	\end{equation}
\end{itemize}

Трудоёмкость в лучшем случае, который наступает при случайном массиве (\ref{for:quick_best}):
\begin{equation}
	\label{for:quick_best}
	f_{best} = (2 + 12(N) * log_2(N)) \approx 12 * N * log_2(N) = O(N*log_2(N))
\end{equation}

Трудоёмкость в худшем случае. В худшем случае, например, отсортированном массиве, глубина стека вызова функции равна N (\ref{for:quick_worst}):
\begin{equation}
	\label{for:quick_worst}
	f_{worst} = (2 + 6(N) * N) \approx  6 * N ^ 2 = O(N ^ 2)
\end{equation}





\subsection{Алгоритм сортировки подсчетом}

Обозначим диапазон значений сортируемого массива за $M$. Тогда трудоёмкость сортировки подсчетом состоит из:
\begin{itemize}
	\item первого цикла
	\begin{equation}
		\label{for:count1}
		f_{first} = 2 + 6 * N
	\end{equation}
	\item второго цикла
	\begin{equation}
		\label{for:count2}
		f_{second} = 2 + 6 * M
	\end{equation}
	\item третьего цикла
	\begin{equation}
		\label{for:count3}
		f_{third} = 2 + 8 * N
	\end{equation}
\end{itemize}

Трудоемкость алгоритма сортировки подсчетом зависит от двух параметров: размера массива и диапазона значений. Таким образом, при $N >> M$ трудоемкость будет равна (\ref{fincount1}):
\begin{equation}
	\label{fincount1}
	f_{1} = 4 + 14 * N \approx 14 * N = O(N)
\end{equation}

Таким образом, при $M >> N$ трудоемкость будет равна (\ref{fincount2}):
\begin{equation}
	\label{fincount2}
	f_{1} = 2 + 6 * M \approx 6 * M = O(M)
\end{equation}


\section*{Вывод}

Были разработаны схемы всех трех алгоритмов сортировки. Для каждого из них были рассчитаны и оценены лучшие и худшие случаи.


