\chapter{Аналитическая часть}
В этом разделе будут представлены описания алгоритмов сортировки перемешиванием, вставками и выбором.

\section{Сортировка перемешиванием}

\textbf{Сортировка перемешиванием} \cite{sheyker} — это разновидность сортировки пузырьком. Отличие в том, что данная сортировка в рамках одной итерации проходит по массиву в обоих направлениях (слева направо и справа налево), тогда как сортировка пузырьком - только в одном направлении (слева направо).

Общие идеи алгоритма:
\begin{itemize}
	\item обход массива слева направо, аналогично пузырьковой - сравнение соседних элементов, меняя их местами, если левое значение больше правого;
	\item обход массива в обратном направлении (справа налево), начиная с элемента, который находится перед последним отсортированным, то есть на этом этапе элементы также сравниваются между собой и меняются местами, чтобы наименьшее значение всегда было слева.
\end{itemize}


\section{Быстрая сортировка}

\textbf{Быстрая сортировка \cite{quick}} — алгоритм сортировки, использующий операцию разбиения массива на две части относительно опорного элемента.

Таким образом, алгоритм быстрой сортировки включает в себя два основных этапа:
\begin{itemize}
	\item разбиение массива относительно опорного элемента;
	\item рекурсивная сортировка каждой части массива.
\end{itemize}

Вне зависимости от того, какой элемент выбран в качестве опорного, массив будет отсортирован, но все же наиболее удачным считается ситуация, когда по обеим сторонам от опорного элемента оказывается примерно равное количество элементов. Если длина какой-то из получившихся в результате разбиения частей превышает один элемент, то для нее нужно рекурсивно выполнить упорядочивание, т. е. повторно запустить алгоритм на каждом из отрезков.


\section{Сортировка подсчетом}

\textbf{Сортировка подсчетом \cite{select}} - один из способов упорядочить массив за линейное время. Применять его можно только для целых чисел, небольшого диапазона, т.к. он требует $O(M)$ дополнительной памяти, где $M$ — ширина диапазона сортируемых чисел. Алгоритм особо эффективен когда требуется отсортировать большое количество чисел, значения которых имеют небольшой разброс.

В сортировке подсчетом элементы массива не сравниваются друг с другом. Эксплуатируется следующая идея:

\begin{enumerate}
	\item требуется отсортировать массив $Array$ из $N$ чисел в диапазоне от 1 до $m$;
	\item подсчитать, сколько раз встречается каждый элемент массива — для этого:
	\begin{enumerate}
		\item создать вспомогательный массив $Counts$ из $m$ счетчиков, заполненный его нулями;
		\item при обходе $Array$, для каждого его элемента увеличить счетчик: $Counts[Value] = Counts[Value] + 1$;
	\end{enumerate}
	\item обойти массив $Counts$, для каждого его $i$-того элемента вывести значение $i$ столько раз, сколько он встретился в исходном массиве. Индекс $i$ при этом соответствует значению числа исходного массива.
\end{enumerate}

\section*{Вывод}

В данной работе стоит задача реализации 3 алгоритмов сортировки, а
именно: перемешиванием, быстрая и подстчетом. Необходимо оценить теоретическую оценку алгоритмов и проверить ее экспериментально.



