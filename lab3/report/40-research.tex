\chapter{Исследовательская часть}

В данном разделе будут приведены примеры работы программ, постановка эксперимента и сравнительный анализ алгоритмов на основе полученных данных.

\section{Технические характеристики}

Технические характеристики устройства, на котором выполнялось тестирование:

\begin{itemize}
	\item операционная система: Manjaro xfce \cite{ubuntu} Linux \cite{linux} x86\_64;
	\item память: 8 Гб;
	\item мобильный процессор AMD Ryzen™ 7 3700U @ 2.3Гц \cite{intel}.
\end{itemize}

Тестирование проводилось на ноутбуке, включенном в сеть электропитания. Во время тестирования ноутбук был нагружен только встроенными приложениями окружения, а также непосредственно системой тестирования.

\section{Демонстрация работы программы}

На рисунке \ref{img:primer} представлен результат работы программы.

\img{60mm}{primer}{Пример работы программы}
\FloatBarrier

\section{Время выполнения алгоритмов}

Алгоритмы тестировались при помощи функции process\_time() из библиотеки time языка Python. Данная функция всегда возвращает значения времени, а имеено сумму системного и пользовательского процессорного времени текущего процессора, типа float в секундах.

Контрольная точка возвращаемого значения не определна, поэтому допустима только разница между результатами последовательных вызовов.

Результаты замеров приведены в таблицах \ref{tbl:best}, \ref{tbl:wor} и \ref{tbl:random}.

На рисунках \ref{img:sorted}, \ref{img:revers} и \ref{img:rand}, приведены графики зависимостей времени работы алгоритмов сортировки от размеров массивов на отсортированных, обратно отсортированных и случайных данных.


\img{90mm}{sorted}{Зависимость времени работы алгоритма сортировки от размера отсортированного массива (мск)}

\FloatBarrier

\begin{table}[h]
	\begin{center}
		\caption{Время работы алгоритмов сортировки на отсортированных данных (мск)}
		\label{tbl:best}
		\begin{tabular}{|c|c|c|c|}
			\hline
			 Размер & Посчетом &  Быстрая &  Шейкер \\
			\hline
			100 & 32.256 & 739.074 & 9.280\\
			\hline
			200 & 77.017 & 2824.264 & 21.749\\
			\hline
			300 & 144.730 & 7463.536 & 40.958 \\
			\hline
			400 & 238.323 & 16039.033 & 68.063 \\
			\hline
			500 & 357.503 & 29838.043 & 103.308 \\
			\hline
			600 & 502.003 & 50185.162 & 146.657 \\
			\hline
			700 & 671.318 & 78380.342 & 198.029 \\
			\hline
			800 & 864.330 & 115650.530 & 257.407 \\
			\hline
			900 & 1084.393 & 163551.843 & 325.160 \\
			\hline
		\end{tabular}
	\end{center}
\end{table}
\FloatBarrier


\img{90mm}{revers}{Зависимость времени работы алгоритма сортировки от размера массива, отсортированного в обратном порядке (мск)}

\FloatBarrier

\begin{table}[h]
	\begin{center}
		\caption{ Время работы алгоритмов сортировки на обратно	отсортированных данных (мск)}
		\label{tbl:wor}
		\begin{tabular}{|c|c|c|c|}
			\hline
			Размер & Посчетом &  Быстрая &  Шейкер \\
			\hline
			100 & 24.156 & 393.429 & 802.331 \\
			\hline
			200 & 66.354 & 1703.255 & 3670.348 \\
			\hline
			300 & 132.379 & 4689.759 & 10237.942 \\
			\hline
			400 & 222.335 & 10350.337 & 22936.469 \\
			\hline
			500 & 337.700 & 19849.474 & 44660.077 \\
			\hline
			600 & 480.300 & 34246.585 & 77731.476 \\
			\hline
			700 & 648.921 & 54421.607 & 124507.850 \\
			\hline
			800 & 843.705 & 81308.332 & 187051.390 \\
			\hline
			900 & 1066.708 & 115931.485 & 267315.687 \\
			\hline
		\end{tabular}
	\end{center}
	
\end{table}
\FloatBarrier

\img{90mm}{rand}{Зависимость времени работы алгоритма сортировки от размера массива, заполненного в случайном порядке (мск)}

\FloatBarrier

\begin{table}[h]
	\begin{center}
		\caption{ Время работы алгоритмов сортировки на случайных данных (мск)}
		\label{tbl:random}
		\begin{tabular}{|c|c|c|c|}
			\hline
			Размер & Посчетом &  Быстрая &  Шейкер \\
			\hline
			100 & 175.389 & 143.153 & 803.232 \\
			\hline
			200 & 288.235 & 315.064 & 2560.598 \\
			\hline
			300 & 418.869 & 601.610 & 6584.375 \\
			\hline
			400 & 571.802 & 1034.448 & 14517.472 \\
			\hline
			500 & 741.793 & 1611.465 & 27777.835 \\
			\hline
			600 & 929.885 & 2341.030 & 47499.079 \\
			\hline
			700 & 1133.956 & 3225.462 & 75728.871 \\
			\hline
			800 & 1353.892 & 4257.337 & 113124.555 \\
			\hline
			900 & 1591.011 & 5451.507 & 160881.977 \\
			\hline
		\end{tabular}
	\end{center}

\end{table}
\FloatBarrier

\section*{Вывод}

Алгоритм сортировки посчетом работает лучше остальных двух во всех трех случаях, что произошло благодаря небольшому диапазону значений генерируемых массивов -- не более 1000. Это привело к линейной трудоемкости данного алгоритма.


