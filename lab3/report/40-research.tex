\chapter{Исследовательская часть}

В данном разделе будут приведены примеры работы программ, постановка эксперимента и сравнительный анализ алгоритмов на основе полученных данных.

\section{Технические характеристики}

Технические характеристики устройства, на котором выполнялось тестирование:

\begin{itemize}
	\item операционная система: Manjaro xfce \cite{ubuntu} Linux \cite{linux} x86\_64;
	\item память: 8 Гб;
	\item мобильный процессор AMD Ryzen™ 7 3700U @ 2.3 ГГц \cite{intel}.
\end{itemize}

Тестирование проводилось на ноутбуке, включенном в сеть электропитания. Во время тестирования ноутбук был нагружен только встроенными приложениями окружения, а также непосредственно системой тестирования.

\section{Демонстрация работы программы}

На рисунке \ref{img:primer} представлен результат работы программы.

\img{60mm}{primer}{Пример работы программы}
\FloatBarrier

\section{Время выполнения реализации алгоритмов}

Время работы реализации алгоритмов измерялось при помощи функции process\_time() из библиотеки time языка Python. Данная функция всегда возвращает значения времени, а имеено сумму системного и пользовательского процессорного времени текущего процессора, типа float в секундах.

Контрольная точка возвращаемого значения не определена, поэтому допустима только разница между результатами последовательных вызовов.

Результаты замеров приведены в таблицах \ref{tbl:best}, \ref{tbl:wor} и \ref{tbl:random}.

На рисунках \ref{img:sorted}, \ref{img:revers} и \ref{img:rand}, приведены графики зависимостей времени работы алгоритмов сортировки от размеров массивов на отсортированных, обратно отсортированных и случайных данных.


\img{90mm}{sorted}{Зависимость времени выполнения реализации алгоритма сортировки от размера отсортированного массива (мкс)}

\FloatBarrier

\begin{table}[h]
	\begin{center}
		\caption{Время выполнения реализации алгоритмов сортировки на отсортированных данных (мкс)}
		\label{tbl:best}
		\begin{tabular}{|c|c|c|c|}
			\hline
			 Размер & Подсчетом &  Быстрая &  Шейкер \\
			\hline
			100 &   23.145 &  545.444 &    6.838\\
			\hline
			200 &   65.845 & 2541.844 &   19.117\\
			\hline
			300 &  131.516 & 6997.099 &   38.015\\
			\hline
			400 &  221.289 & 15206.390 &   64.844\\
			\hline
			500 &  337.287 & 28752.306 &   99.571\\
			\hline
			600 &  477.355 & 48464.394 &  142.147\\
			\hline
			700 &  644.714 & 75756.450 &  192.762\\
			\hline
			800 &  836.860 & 112060.297 &  251.221\\
			\hline
			900 & 1054.386 & 159101.962 &  318.163\\
			\hline
		\end{tabular}
	\end{center}
\end{table}
\FloatBarrier


\img{90mm}{revers}{Зависимость времени выполнения реализации алгоритма сортировки от размера массива, отсортированного в обратном порядке (мкс)}

\FloatBarrier

\begin{table}[h]
	\begin{center}
		\caption{ Время выполнения реализации алгоритмов сортировки на обратно	отсортированных данных (мкс)}
		\label{tbl:wor}
		\begin{tabular}{|c|c|c|c|}
			\hline
			Размер & Подсчетом &  Быстрая &  Шейкер \\
			\hline
			100 &   33.426 &  530.626 & 1078.104\\
			\hline
			200 &   76.742 & 1868.477 & 3944.645\\
			\hline
			300 &  143.454 & 4870.720 & 10498.472\\
			\hline
			400 &  234.621 & 10557.311 & 23275.374\\
			\hline
			500 &  350.191 & 20097.290 & 45212.845\\
			\hline
			600 &  489.927 & 34519.663 & 78491.104\\
			\hline
			700 &  653.540 & 54741.962 & 125775.118\\
			\hline
			800 &  845.667 & 81818.746 & 189079.515\\
			\hline
			900 & 1062.585 & 116464.102 & 269914.922\\
			\hline
		\end{tabular}
	\end{center}
	
\end{table}
\FloatBarrier

\img{90mm}{rand}{Зависимость времени выполнения реализации алгоритма сортировки от размера массива, заполненного в случайном порядке (мкс)}

\FloatBarrier

\begin{table}[h]
	\begin{center}
		\caption{ Время выполнения реализации алгоритмов сортировки на случайных данных (мкс)}
		\label{tbl:random}
		\begin{tabular}{|c|c|c|c|}
			\hline
			Размер & Подсчетом &  Быстрая &  Шейкер \\
			\hline
			100 &  129.406 &  102.688 &  588.341\\
			\hline
			200 &  243.097 &  277.647 & 2339.547\\
			\hline
			300 &  371.899 &  565.186 & 6360.275\\
			\hline
			400 &  521.326 &  988.754 & 14302.226\\
			\hline
			500 &  689.655 & 1555.392 & 27568.101\\
			\hline
			600 &  875.418 & 2270.616 & 47477.725\\
			\hline
			700 & 1077.845 & 3148.489 & 75123.545\\
			\hline
			800 & 1296.092 & 4189.439 & 112105.260\\
			\hline
			900 & 1530.593 & 5370.045 & 159892.601\\
			\hline
		\end{tabular}
	\end{center}

\end{table}
\FloatBarrier

\section*{Вывод}

Реализация алгоритма сортировки подсчетом работает быстрее остальных двух во всех трех случаях, что произошло благодаря небольшому диапазону значений генерируемых массивов --- не более 1000. Это привело к линейному характеру зависимости трудоемкости данного алгоритма от размера массива и линейному от мощность диапазона значений в массиве.


