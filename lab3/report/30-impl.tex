\chapter{Технологическая часть}

В данном разделе будут приведены требования к программному обеспечению, средства реализации и листинги кода.

\section{Требования к ПО}

К программе предъявляется ряд требований:
\begin{itemize}
	\item на вход подаётся массив сравнимых элементов (целые числа);
	\item на выходе — тот же массив, но в отсортированном порядке.
\end{itemize}

\section{Средства реализации}

В качестве языка программирования для реализации данной лабораторной работы был выбран язык программирования Python \cite{pythonlang}. 

В данном языке есть все требующиеся для данной лабораторной инструменты разработки. 

Процессорное время работы реализаций алгоритмов было замерено с помощью функции process\_time() из библиотеки time \cite{pythonlangtime}

\section{Сведения о модулях программы}
Программа состоит из трех модулей:
\begin{enumerate}
	\item main.py - главный файл программы, в котором располагается меню;
	\item algos.py - файд программы, в котором располагаются коды всех алгоритмов;
	\item test.py - файл с замерами времени для графического изображения результата.
\end{enumerate}


\section{Реализация алгоритмов}
В листингах \ref{lst:quick_sort}, \ref{lst:shaker_sort}, \ref{lst:count_sort} представлены реализации алгоритмов сортировок (перемешиванием, вставками и выбором).

\begin{lstlisting}[label=lst:quick_sort,caption=Реализация алгоритма быстрой сортировки]
def partition(arr, low, high):
	p = arr[high]
	i = low - 1
	for j in range(low, high):
		if arr[j] <= p:
			i += 1
			arr[i], arr[j] = arr[j], arr[i]
			arr[i + 1], arr[high] = arr[high], arr[i + 1]
	return i + 1

def quick(arr, low, high):
	if low < high:
		pi = partition(arr, low, high)
		quick_sort(arr, low, pi - 1)
		quick_sort(arr, pi + 1, high)

def quick_sort(arr):
	quick(arr, 0, len(arr) - 1)
\end{lstlisting}

\begin{lstlisting}[label=lst:shaker_sort,caption=Реализация алгоритма сортировки перемешиванием]
def shaker_sort(arr):
	swapped = True
	st = 0
	end = len(arr) - 1
	while (swapped):
		swapped = False
		for i in range(st, end):
			if (arr[i] > arr[i + 1]):
				arr[i], arr[i + 1] = arr[i + 1], arr[i]
				swapped = True
		if (not swapped):
			break
		swapped = False
		end -= 1
		for i in range(end - 1, st - 1, -1):
			if (arr[i] > arr[i + 1]):
				arr[i], arr[i + 1] = arr[i + 1], arr[i]
				swapped = True
		st += 1
	return arr
\end{lstlisting}

\begin{lstlisting}[label=lst:count_sort,caption=Реализация алгоритма сортировки подсчетом]
def counting_sort_help(arr, largest):
	c = [0]*(largest + 1)
	for i in range(len(arr)):
		c[arr[i]] += 1
	
	c[0] -= 1 
	for i in range(1, largest + 1):
		c[i] += c[i - 1]
	
	result = [0]*len(arr)
	
	for x in reversed(arr):
		result[c[x]] = x
		c[x] -= 1
		
	return result

def counting_sort(arr):
	largest = max(arr)
return counting_sort_help(arr, largest)
\end{lstlisting}

\section{Функциональные тесты}

В таблице \ref{tbl:functional_test} приведены тесты для функций, реализующих алгоритмы сортировки. Тесты пройдены успешно.


\begin{table}[h]
	\begin{center}
		\caption{\label{tbl:functional_test} Функциональные тесты}
		\begin{tabular}{|c|c|c|}
			\hline
			Входной массив & Ожидаемый результат & Результат \\ 
			\hline
			$[1,2,3,4]$ & $[1,2,3,4]$  & $[1,2,3,4]$\\
			$[5,4,3,2,1]$  & $[1,2,3,4,5]$ & $[1,2,3,4,5]$\\
			$[3,2,-5,0,1]$  & $[-5,0,0,2,3]$  & $[-5,0,0,2,3]$\\
			$[4]$  & $[4]$  & $[4]$\\
			$[]$  & $[]$  & $[]$\\
			\hline
		\end{tabular}
	\end{center}
\end{table}


\section*{Вывод}

Были реализованы все три алгоритма сортировки. Для каждого из них были рассчитаны и оценены лучшие и худшие случаи.
