\chapter{Технологическая часть}

В данном разделе будут приведены требования к программному обеспечению, средства реализации и листинги кода.

\section{Требования к ПО}

К программе предъявляется ряд требований:
\begin{itemize}
	\item[-] на вход подается количество умножений;
	\item[-] используется параллелизм программы;
	\item[-] возможно измерение реального времени;
\end{itemize}

\section{Средства реализации}

В качестве языка программирования для реализации данной лабораторной работы был выбран язык программирования Си++ \cite{pythonlang}. 

В данном языке есть все требующиеся для данной лабораторной инструменты разработки. 

Процессорное время работы реализаций алгоритмов было замерено с помощью функции high\_resolution\_clock::now() из библиотеки <chrono> \cite{pythonlangtime}.

\section{Реализация алгоритмов}
В листинге \ref{lst:quick_sort} представлена реализация алгоритма умножения разреженных матриц.

\begin{lstlisting}[label=lst:quick_sort,caption=Реализация последовательного алгоритма умножения разреженных матриц]
matr *mutl(matr *a, matr *b, const int m)
{
	matr *new_matr = new(matr); // 1
	new_matr->rows_index.push_back(0);
	int col = 0;
	for (int i = 1; i < int(a->rows_index.size()); i++) // 2
	{
		int start = a->rows_index[i-1];
		int end = a->rows_index[i];
		int *prom = (int *)calloc(m, sizeof(int));
		for (int j = start; j < end; j++) { // 3
			int x = a->elements[j];
			for (int k = 0; k < int(b->elements.size()); k++) {
				int row;
				for (row = 0; row < int(b->rows_index.size() - 1); ++row) {
					if (b->rows_index[row] <= k && b->rows_index[row + 1] > k)
					break;
				}
				if (row == a->columns[j])
				prom[b->columns[k]] += x * b->elements[k]; // 3.a
			}
		}
		for (int f = 0; f < m; f++) // 3.b
		if (prom[f]) {
			new_matr->elements.push_back(prom[f]);
			new_matr->columns.push_back(f);
			col++;
		}
		new_matr->rows_index.push_back(col); // 3.c
		free(prom);
	}
	return new_matr; // 4
}
\end{lstlisting}

В листинге \ref{lst:shaker_sort} представлена реализация синхронного конвейера.

\begin{lstlisting}[label=lst:shaker_sort,caption=Реализация функции параллелизации умножения разреженных матриц]

pipeTask* gener(pipeTask* pt) {
	pt->start_generate = chrono::high_resolution_clock::now();
	
	auto k1 = generate();
	pt->m1 = get_csrrepresent_m(k1);
	k1 = generate();
	pt->m2 = get_csrrepresent_m(k1);
	
	pt->stop_generate = chrono::high_resolution_clock::now();
	
	return pt;
}

pipeTask* multic(pipeTask* pt) {
	pt->start_mul = chrono::high_resolution_clock::now();
	
	pt->mrez = mutl((pt->m1), (pt->m2), N);
	
	pt->stop_mul = chrono::high_resolution_clock::now();
	
	return pt;
}

pipeTask* write(pipeTask* pt) {
	pt->start_write = chrono::high_resolution_clock::now();
	
	ofstream out("out.txt");
	
	
	out << '\n';
	for (int j = 0; j < int(pt->mrez->rows_index.size()); j++)
	out << pt->mrez->rows_index[j] << ' ';
	out << '\n';
	for (int j = 0; j < int(pt->mrez->columns.size()); j++)
	out << pt->mrez->columns[j] << ' ';
	out << '\n';
	for (int j = 0; j < int(pt->mrez->elements.size()); j++)
	out << pt->mrez->elements[j] << ' ';
	out << '\n';
	
	out.close();
	
	pt->stop_write = chrono::high_resolution_clock::now();
	
	return pt;
}


void gener_conv(queue* qu, queue* qu2, int l) {
	int ll = 0;
	while (1) {
		if (qu->queue.size() > 0) {
			auto k = gener(qu->queue[0]);
			qu->queue.erase(qu->queue.begin());
			qu2->queue.push_back(k);
			ll++;
		}
		
		if (ll == l)
		break;            
	}
}

void multic_conv(queue* qu, queue* qu3, int l) {
	int ll = 0;
	while (1) {
		if (qu->queue.size() > 0) {
			auto k = multic(qu->queue[0]);
			qu->queue.erase(qu->queue.begin());
			qu3->queue.push_back(k);
			ll++;
		}
		
		if (ll == l)
		break; 
	}
}

void write_conv(queue* qu, queue* quf, int l) {
	int ll = 0;
	while (1) {
		if (qu->queue.size() > 0) {
			auto k = write(qu->queue[0]);
			qu->queue.erase(qu->queue.begin());
			quf->queue.push_back(k);
			ll++;
		}
		
		if (ll == l)
		break; 
	}
}

queue* pipeline(int count) {
	vector<thread> threads(3);
	
	queue* qu1 = new(queue);
	queue* qu2 = new(queue);
	queue* qu3 = new(queue);
	queue* quf = new(queue);
	
	qu1->stop = false;
	qu2->stop = false;
	qu3->stop = false;
	
	threads[0] = thread(gener_conv, qu1, qu2, count);
	threads[1] = thread(multic_conv, qu2, qu3, count);
	threads[2] = thread(write_conv, qu3, quf, count);
	
	for (int i = 0; i < count; i++) {
		auto pt = new(pipeTask);
		pt->n = i;
		qu1->queue.push_back(pt);
	}
	
	for (auto& thr: threads)
	thr.join();
	
	return quf;
}
}
}
\end{lstlisting}

\section{Функциональные тесты}

В таблице \ref{tbl:functional_test} приведены тесты для функций, реализующих алгоритмы сортировки. Тесты пройдены успешно.

\begin{table}
	\begin{center}
		\captionsetup{singlelinecheck = false, justification=centering}
		\caption{Тестовые случаи}
		\begin{tabular}{l|l|l|l}
			N & Матрица 1 & Матрица 2 & Результат \\ 
			\hline
			\vspace{2mm}
			\vspace{2mm}
			1 & $\begin{pmatrix}
				1 & 1 & 1\\
				5 & 5 & 5\\
				2 & 2 & 2
			\end{pmatrix}$ &
			$\begin{pmatrix}
				1 \\
				1 \\
				1 
			\end{pmatrix}$ &
			$\begin{pmatrix}
				3 \\
				15 \\
				6
			\end{pmatrix}$ \\
			\vspace{2mm}
			\vspace{2mm}
			2 & $\begin{pmatrix}
				1 & 1 & 1
			\end{pmatrix}$ &
			$\begin{pmatrix}
				1 \\
				1 \\
				1
			\end{pmatrix}$ &
			$\begin{pmatrix}
				1 & 1 & 1\\
				1 & 1 & 1 \\
				1 & 1 & 1
			\end{pmatrix}$ \\
			\vspace{2mm}
			\vspace{2mm}
			3 & $\begin{pmatrix}
				1 & 1 \\
				1 & 1
			\end{pmatrix}$ &
			$\begin{pmatrix}
				1 & 1\\
				1 & 1
			\end{pmatrix}$ &
			$\begin{pmatrix}
				2 & 2\\
				2 & 2
			\end{pmatrix}$ \\
			\vspace{2mm}
			\vspace{2mm}
			4 & $\begin{pmatrix}
				2
			\end{pmatrix}$ &
			$\begin{pmatrix}
				2
			\end{pmatrix}$ &
			$\begin{pmatrix}
				4
			\end{pmatrix}$ \\
			\vspace{2mm}
			\vspace{2mm}
			5 & $\begin{pmatrix}
				1 & -2 & 3\\
				1 & 2 & 3\\
				1 & 2 & 3
			\end{pmatrix}$ &
			$\begin{pmatrix}
				-1 & 2 & 3\\
				1 & 2 & 3\\
				1 & 2 & 3
			\end{pmatrix}$ &
			$\begin{pmatrix}
				0 & 4 & 6\\
				4 & 12 & 18\\
				4 & 12 & 18
			\end{pmatrix}$
		\end{tabular}
	\label{tbl:functional_test}
	\end{center}
\end{table}

\FloatBarrier


\section*{Вывод}

Написано и протестировано программное обеспечение для поставленной задачи.
