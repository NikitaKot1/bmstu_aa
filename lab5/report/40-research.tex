\chapter{Исследовательская часть}

В данном разделе будут приведены примеры работы программ, постановка эксперимента и сравнительный анализ алгоритмов на основе полученных данных.

\section{Технические характеристики}

Технические характеристики устройства, на котором выполнялось тестирование:

\begin{itemize}
	\item операционная система: Manjaro xfce \cite{ubuntu} Linux \cite{linux} x86\_64;
	\item память --- 8 Гб;
	\item мобильный процессор AMD Ryzen™ 7 3700U @ 2.3 ГГц \cite{intel}.
\end{itemize}

Тестирование проводилось на ноутбуке, включенном в сеть электропитания. Во время тестирования ноутбук был нагружен только встроенными приложениями окружения, а также непосредственно системой тестирования.

\section{Демонстрация работы программы}

На рисунке \ref{img:primer} представлен результат работы программы.

\img{140mm}{primer}{Пример работы программы}
\FloatBarrier

\section{Время выполнения реализации алгоритмов}

Время работы реализации алгоритмов измерялось при помощи функции chrono::high\_resolution\_clock::now() из библиотеки <chrono> языка С++.

\begin{table}[ht!]
	\begin{center}
		\captionsetup{justification=raggedright,singlelinecheck=off}
		\label{tbl:best}
		\begin{tabular}{|c|c|c|c|c|c|c|}
			\hline
			\multirow{3}{*}{№} & \multicolumn{6}{c|}{Начало обработки заявки} \\ \cline{2-7} 
			& \multicolumn{3}{c|}{Параллельно, мс.} & \multicolumn{3}{c|}{Синхронно, мс.} \\ \cline{2-7} 
			& 1 & 2 & 3 & 1 & 2 & 3 \\ \hline
			1   & 0          & 3050      & 435427     & 0          & 4546     & 455827      \\
			2   & 3050      & 435427     &   828247   &   456041    &  457950    &  865352  \\
			3   &   8130    &  828247    & 1228188     &   865554    &  867585   &  1278679    \\
			4   &   12261   &   1228188   &   1638834   &  1278873    &  1281023    &  1680580   \\
			5   &   14427   &  1638834    &  2039991    &  1680808   &  1682876    &  2092687    \\
			6   &   16579   &  2039991    &  2441431    &  2092876    &  2094858    &  2497214    \\
			7   &   18724   &  2441431    &   2858998   &  2497453    &  2499488   &   2907603   \\
			8   &   20974   &   2858998   &  3242625    &  2907837   &   2909908   &   3313635   \\
			\hline
		\end{tabular}
	\caption{Замеры времени работы на очереди размером 8}
	\end{center}
\end{table}

\FloatBarrier

Из таблицы можно сделать вывод, что рапараллеленый  конвейер выполняет работу на $10\%$ быстрее, чем последовательный.

\section*{Вывод}

В данном разделе были сравнены алгоритмы по времени. Выявлено, что конвейерная обработка быстрее последовательной на $10\%$.