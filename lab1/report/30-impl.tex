\chapter{Технологическая часть}

В данном разделе будут приведены требования к программному обеспечению, средства реализации и листинги кода.

\section{Требования к ПО}

К программе предъявляется ряд требований:
\begin{itemize}
	\item реализованны все алгоритмы, соответствующие варианту;
	\item на вход подаются две строки на русском или английском языке в любом регистре;
	\item программа выдает искомое расстояние для выбранного метода (выбранных методов) и матрицы расстояний для матричных реализаций.
\end{itemize}

\section{Средства реализации}

В качестве языка программирования для реализации данной лабораторной работы был выбран язык программирования Golang \cite{pythonlang}. 

В данном языке есть все требующиеся для данной лабораторной инструменты разработки. 

Время работы алгоритмов будет измеряться с помощью команды time в Linux \cite{pythonlangtime}.

\section{Сведения о модулях программы}
Программа состоит из трех модулей:
\begin{enumerate}[label={\arabic*)}]
	\item main.go -- главный файл программы;
	\item algos.go -- файл программы, в котором располагаются коды всех алгоритмов;
	\item sub.go -- файл программы, в котором располагаются вспомогательные функции.
\end{enumerate}


\section{Листинг кода}

 В листингах \ref{lst:lev_rec}, \ref{lst:lev_mat}, \ref{lst:lev_rec_mat}, \ref{lst:dlev} приведены реализации алгоритмов нахождения расстояния Левенштейна и Дамерау--Левенштейна.

\begin{lstlisting}[label=lst:lev_rec,caption=Функция нахождения расстояния Дамерау--Левенштейна с использованием рекурсии]
func dam_lev_recursive(str1, str2 string, output bool) int {
	min_ret := 0
	n := len(str1)
	m := len(str2)
	if n == 0 {
		return m
	}
	if m == 0 {
		return n
	}
	change := 0
	if str1[n-1] != str2[m-1] {
		change += 1
	}
	if n > 1 && m > 1 && str1[n-1] == str2[m-2] && str1[n-2] == str2[m-1] {
		min_ret = MinOf(dam_lev_recursive(str1[:n-1], str2, output)+1,
		dam_lev_recursive(str1, str2[:m-1], output)+1,
		dam_lev_recursive(str1[:n-1], str2[:m-1], output)+change,
		dam_lev_recursive(str1[:n-2], str2[:m-2], output)+1)
	} else {
		min_ret = MinOf(dam_lev_recursive(str1[:n-1], str2, output)+1,
		dam_lev_recursive(str1, str2[:m-1], output)+1,
		dam_lev_recursive(str1[:n-1], str2[:m-1], output)+change)
	}
	return min_ret
}
	
\end{lstlisting}

\begin{lstlisting}[label=lst:lev_mat,caption=Функция нахождения расстояния Дамерау--Левенштейна с использованием матрицы.]
func dam_lev_matrix(str1, str2 string, output bool) int {
	min_ret := 0
	n := len(str1)
	m := len(str2)
	matr := createMatr(n+1, m+1)
	for i := 1; i < n+1; i++ {
		for j := 1; j < m+1; j++ {
			add, delete, change := matr[i-1][j]+1, matr[i][j-1]+1, matr[i-1][j-1]
			
			if str1[i-1] != str2[j-1] {
				change += 1
			}
			
			matr[i][j] = MinOf(add, delete, change)
			
			if i > 1 && j > 1 && str1[i-1] == str2[j-2] && str1[i-2] == str2[j-1] {
				matr[i][j] = MinOf(matr[i][j], matr[i-2][j-2]+1)
			}
		}
	}
	return min_ret
}
	
\end{lstlisting}

\begin{lstlisting}[label=lst:lev_rec_mat,caption=Функция нахождения расстояния Дамерау--Левенштейна с использованием рекурсии с кешем.]
func dam_lev_rec_cash_help(str1, str2 string, matr MatrInt) {
	len1 := len(str1)
	len2 := len(str2)
	
	if len1 == 0 {
		matr[len1][len2] = len2
	} else if len2 == 0 {
		matr[len1][len2] = len1
	} else {
		//insert
		if matr[len1][len2-1] == INF {
			dam_lev_rec_cash_help(str1, str2[:len2-1], matr)
		}
		//delete
		if matr[len1-1][len2] == INF {
			dam_lev_rec_cash_help(str1[:len1-1], str2, matr)
		}
		//replase
		if matr[len1-1][len2-1] == INF {
			dam_lev_rec_cash_help(str1[:len1-1], str2[:len2-1], matr)
		}
		
		change := 0
		if str1[len1-1] != str2[len2-1] {
			change += 1
		}
		
		matr[len1][len2] = MinOf(matr[len1][len2-1]+1,
		matr[len1-1][len2]+1,
		matr[len1-1][len2-1]+change)
		
		if len1 > 1 && len2 > 1 && str1[len1-1] == str2[len2-2] && str1[len1-2] == str2[len2-1] {
			matr[len1][len2] = MinOf(matr[len1][len2], matr[len1-2][len2-2]+1)
		}
	}
}

func dam_lev_rec_cash(str1, str2 string, output bool) int {
	n := len(str1)
	m := len(str2)
	
	matr := createInfMatrix(n+1, m+1)
	dam_lev_rec_cash_help(str1, str2, matr)
	
	if output {
		printMatr(matr, n+1, m+1)
	}
	
	return matr[n][m]
}
\end{lstlisting}

\begin{lstlisting}[label=lst:dlev,caption=Функция нахождения расстояния Левенштейна с использованием итераций.]
func lev_matrix(str1, str2 string, output bool) int {
	n := len(str1)
	m := len(str2)
	
	matr := createMatr(n+1, m+1)
	for i := 1; i < n+1; i++ {
		for j := 1; j < m+1; j++ {
			add, delete, change := matr[i-1][j]+1, matr[i][j-1]+1, matr[i-1][j-1]
			
			if str1[i-1] != str2[j-1] {
				change += 1
			}
			
			matr[i][j] = MinOf(add, delete, change)
		}
	}
	
	if output {
		printMatr(matr, n+1, m+1)
	}
	
	return matr[n][m]
}
\end{lstlisting}

\section{Функциональные тесты}
В таблице \ref{tabular:functional_test} приведены функциональные тесты для алгоритмов вычисления расстояния Левенштейна (в таблице столбец подписан "Левенштейн") и Дамерау — Левенштейна (в таблице - "Дамерау-Л."). Все тесты пройдены успешно.


\begin{table}[h]
	\begin{center}
		\caption{\label{tabular:functional_test} Функциональные тесты}
		\begin{tabular}{|c|c|c|c|c|}
			\hline
			& \multicolumn{2}{c|}{Входные данные} & \multicolumn{2}{c|}{Ожидаемый результат} \\
			\hline
			№&Строка 1&Строка 2&Левенштейн&Дамерау-Л. \\
			\hline
			1&скат&кот&2&2 \\
			\hline
			2&машина&малина&1&1 \\
			\hline
			3&дворик&доврик&2&1 \\
			\hline
			4&"пустая строка"&университет&11&11 \\
			\hline
			5&сентябрь&"пустая строка"&8&8 \\
			\hline
			8&тело&телодвижение&8&8 \\
			\hline
			9&ноутбук&планшет&7&7 \\
			\hline
			10&глина&малина&2&2 \\
			\hline
			11&рекурсия&ркерусия&3&2 \\
			\hline
			12&браузер&баурзер&2&2 \\
			\hline
			13&bring&brought&4&4 \\
			\hline
			14&moment&minute&4&4 \\ 
			\hline
			15&person&eye&5&5 \\
			\hline
			16&week&weekend&3&3 \\
			\hline 
			17&city&town&4&4 \\
			\hline
		\end{tabular}
	\end{center}
\end{table}


\section*{Вывод}

Были разработаны и протестированы алгоритмы: нахождения расстояния Дамерау -- Левенштейна рекурсивно, с заполнением матрицы и рекурсивно с заполнением матрицы, а также нахождения расстояния Левенштейна с заполнением матрицы.
