\chapter{Конструкторская часть}
В этом разделе будут приведены требования к вводу и программе, а также схемы алгоритмов нахождения расстояний Левенштейна и Дамерау-Левенштейна.

\section{Требования к вводу}
К вводу программы должны быть предъявлены следующие требования.
\begin{enumerate}
	\item На вход подаются две строки.
	\item Буквы верхнего и нижнего регистров считаются различными.
\end{enumerate}

\section{Требования к программе}
К программе должны быть предъявлены следующие требования.
\begin{enumerate}
	\item Две пустые строки - корректный ввод, программа не должна аварийно завершаться.
	\item На выход программа должна вывести число - расстояние Левенштейна (Дамерау-Левенштейна), матрицу при необходимости.
\end{enumerate}

\section{Алгоритмы нахождения расстояния Дамерау -- Левенштейна}

На рисунке \ref{img:dam_lev_rec_wo_cash} приведена схема рекурсивного алгоритма нахождения расстояния Дамерау -- Левенштейна.

\img{180mm}{dam_lev_rec_wo_cash}{Схема рекурсивного алгоритма нахождения расстояния Дамерау -- Левенштейна}

\FloatBarrier

На рисунке \ref{img:dam_lev} приведена схема алгоритма нахождения расстояния Дамерау -- Левенштейна с заполнением матрицы.

\img{220mm}{dam_lev}{Схема алгоритма нахождения расстояния Дамерау -- Левенштейна с заполнением матрицы}

\FloatBarrier

На рисунке \ref{img:dam_lev_rec_cash} приведена схема рекурсивного алгоритма нахождения расстояния Дамерау -- Левенштейна с использованием кеша в виде матрицы.

\img{160mm}{dam_lev_rec_cash}{Схема рекурсивного алгоритма нахождения расстояния Дамерау -- Левенштейна с использованием кеша в виде матрицы}

\FloatBarrier

\section{Алгоритм нахождения расстояния Левенштейна}

На рисунке \ref{img:lev} приведена схема рекурсивного алгоритма нахождения расстояния Левенштейна.

\img{220mm}{lev}{Схема алгоритма нахождения расстояния Левенштейна с заполнением матрицы}

\FloatBarrier



\section{Вывод}

Перечислены требования к вводу и программе, а также на основе теоретических данных, полученных из аналитического раздела, были построены схемы требуемых алгоритмов.


