\chapter*{Заключение}
\addcontentsline{toc}{chapter}{Заключение}

В ходе выполнения лабораторной работы была достигнута цель работа: были разработаны алгоритмы нахождения расстояний Левенштейна и Дамерау--Левенштейна.

Все поставленные задачи решены:

\begin{itemize}
    \item реализовать алгоритмы нахождения расстояний Левенштейна и Дамерау -- Левенштейна;
	\item реализованы алгоритмы поиска расстояния Дамерау -- Левенштейна с заполнением матрицы, с использованием рекурсии и с помощью рекурсивного заполнения матрицы (рекурсивный с использованием кеша);
	\item реализован алгоритм поиска расстояния Левенштейна с использованием матрицы;
	\item проведен сравнительный анализ линейной и рекурсивной реализаций алгоритмов определения расстояния между строками по затрачиваемым ресурсам (времени и памяти);
	\item проведено экспериментальное подтверждение различий во временной эффективности рекурсивной и нерекурсивной реализаций алгоритмов при помощи разработанного программного обеспечения на материале замеров процессорного времени выполнения реализации на различных длинах строк;
	\item подготовлен отчет о лабораторной работе.
\end{itemize}

Экспериментально было подтверждено различие во временной эффективности рекурсивной и нерекурсивной реализаций выбранного алгоритма определения между строками при помощи разработанного программного обеспечения на материале замеров процессорного времени выполнения реализаций на различных длин строк.

По итогу исследований было выявлено, что матричная реализация алгоритмов нахождения расстояний заметно выигрывает по времени при росте строк, но проигрывает по количеству затрачиваемой памяти.
